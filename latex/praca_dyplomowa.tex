\documentclass[12pt, a4paper, oneside, draft]{report}
\usepackage[utf8]{inputenc}
\usepackage[T1]{fontenc}
\usepackage{polski}
\usepackage{graphicx}
\usepackage{listings}
\usepackage{cite}



\usepackage{mathtools, amsthm, amssymb}
\mathtoolsset{showonlyrefs, mathic}

\newtheorem{theorem}{Twierdzenie}
\newtheorem{lemat}{Lemat}
\newtheorem{uwaga}{Uwaga}
\newtheorem{przyklad}{Przykład}
\newtheorem{wniosek}{Wniosek}

\newcommand{\mychapter}[2]{
	\setcounter{chapter}{#1}
	\setcounter{section}{0}
	\chapter*{#2}
	\addcontentsline{toc}{chapter}{#2}
}


\author{Paweł Budzyński}
\title{Wykorzystanie sieci neuronowych w problemie wykrywania uszkodzeń lokalnych w maszynach górniczych}
\date{28.11.2018 r.}

\begin{document}
	\maketitle
	\tableofcontents
	%\mychapter{1}{Wstęp}
	\chapter{Wstęp}
	
	
	\section{Opis problemu}
	\section{Sieci neuronowe}
	
	
	\chapter{Metodologia}
	
	\section{Wstępne przetwarzanie danych}
	
	\section{Zastosowanie sieci neuronowych}
	
	
	
	\chapter{Wyniki dla danych symulowanych}
	
	\section{Surowy sygnał}
	
	\section{Estymatory}
	
	\chapter{Wnioski}
	
	
	
	\begin{center}
%		\includegraphics[width=12cm]{wykres3.pdf}
	\end{center}
	
	\bibliography{mybib}
	\bibliographystyle{plain}
\end{document}