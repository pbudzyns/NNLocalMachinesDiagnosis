\documentclass[12pt, a4paper, oneside, draft]{article}
\usepackage[left=1in, right=1in]{geometry}
\usepackage[utf8]{inputenc}
\usepackage[T1]{fontenc}
\usepackage{polski}
\usepackage{graphicx}
\usepackage{listings}
\usepackage{cite}
%\usepackage[numbers,sort&compress]{natbib}
\usepackage{notoccite}



\usepackage{mathtools, amsthm, amssymb}
\mathtoolsset{showonlyrefs, mathic}

\newtheorem{theorem}{Twierdzenie}
\newtheorem{lemat}{Lemat}
\newtheorem{uwaga}{Uwaga}
\newtheorem{przyklad}{Przykład}
\newtheorem{wniosek}{Wniosek}

\newcommand{\mychapter}[2]{
	\setcounter{chapter}{#1}
	\setcounter{section}{0}
	\chapter*{#2}
	\addcontentsline{toc}{chapter}{#2}
}


\author{Paweł Budzyński \\ Wydział Matematyki, Politechnika Wrocławska}
\title{\textbf{Wykorzystanie sieci neuronowych w problemie wykrywania uszkodzeń lokalnych w maszynach górniczych}}
\date{2018 \\ Październik}

\begin{document}
	\maketitle
	%\tableofcontents
	%\mychapter{1}{Wstęp}
	\section{Wstęp}
	Praca ma na celu znalezienie skutecznego zastosowania sieci neuronowych w problemie wykrywania uszkodzeń lokalnych w maszynach górniczych na podstawie sygnału diagnostycznego
	
	\subsection{Opis problemu}
	wykrywanie uszkodzeń lokalnych jest jednym z \cite{Zimroz01}
	Uszkodzenia lokalne powstające w przekładniach kół czerpakowych maszyn górniczych, są to na przykład ubytki, pęknięcia, wykruszone zęby.  Najskutecznejszą obecnie metodą pozyskania sygnału jest diagnostyka wibroakustyczna, okazuje się że uszkodzenia takie objawiają się cyklicznymi pulsacjami w sygnale. Amplituda pulsacji zależy od stopnia uszkodzenia. Z punktu widzenia matematyki detekcja takiego uszkodzenia polega na wykryciu, zazwyczaj niewidocznych na pierwszy rzut oka, pulsacji w sygnale diagnostycznym. 
	
	
	\subsection{Sieci neuronowe}
	Sieci neuronowe są jedną z prężnie roziwjających się technologii nauczanai maszynowego i szeroko wykorzystywane w sztucznej inteligencji,
	pierwsze badania pojawiły się z wynalezieniem perceptronu w 1957 r., zaprezentować taki perceptron,
	Istnieje wiele rodzajów sieci i są używane w wielu zastosowaniach, w pracy najpopularniejszy jest perceptron wielowarstwowy uczony z użyciem algorytmu propagacj wstecznej 
	obecnie sieci znajduja zastosowanie w wielu dziedzinach, poniżej przykłady.
	
	
	%\chapter{Metodologia}
	\section{Metodologia}
	
	\subsection{Wstępne przetwarzanie danych}
	\cite{Python}
	\subsection{Zastosowanie sieci neuronowych}
	
	
	
	\cite{Wylomanska01}
	
	
	
	%\chapter{Wyniki dla danych symulowanych}
	\section{Wyniki dla symulowanych danych}
	Z powodu zapotrzebowania na dużą ilosć danych w procesie uczenia sieci neuronowych dalsza praca opiera się na danych symulowanych. Program generujący sygnał pozwala na dobór szeregu parametrów takich jak częstotliwości, amplitudy, zaszumienie oraz siłę pulsacji.
	\subsection{Przygotowanie danych}
	Przed przystąpieniem do uczenia należało przygotować dane uczące, w tym przypadku jest to zestaw sygnałów bez pulsacji oraz sygnałów z różnymi pulsacjami, poniżej przedstawiono tabelkę szczegółowo opisującą przygotowany zbiór danych.
	\subsection{Surowy sygnał}
	Pierwszym eksperymentem było przetestowanie jak sieć poradzi sobie w przypadku surowych danych. Do nauczania użyto sygnału odpowiadającemu jednej sekundzie odczytu, co przy próbkowaniu na poziomie $x$ daje $w chuj$ próbek.
	\\ Tutaj jebniesz schemat sieci
	\\A tutaj tabelka z wniami 
	
	Eksperyment ten przeprowadzony był bardziej dla porównania niż z żeczywistej potrzeby, użycie surowego sygnału i takiej ilości próbek nie wydaje się był optymalnym rozwiązaniem za to jest kosztowne, dobrą praktyką w dziedzinie nauczania maszynowego jest wstępna obróbka danych i znalezienie sposobu na 
	
	\subsection{Estymatory}
	
	%\chapter{Wnioski}
	\section{Wnioski}
	
	
	
	\begin{center}
%		\includegraphics[width=12cm]{wykres3.pdf}
	\end{center}
	
	
	%\bibliographystyle{unsrt}
	\bibliography{mybib}
\end{document}