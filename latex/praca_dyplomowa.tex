\documentclass[12pt]{article}
\usepackage[utf8]{inputenc}
\usepackage[T1]{fontenc}
\usepackage{polski}
\usepackage{graphicx}
\usepackage{listings}
\usepackage{cite}



\usepackage{mathtools, amsthm, amssymb}
\mathtoolsset{showonlyrefs, mathic}

\newtheorem{theorem}{Twierdzenie}
\newtheorem{lemat}{Lemat}
\newtheorem{uwaga}{Uwaga}
\newtheorem{przyklad}{Przykład}
\newtheorem{wniosek}{Wniosek}


\author{Paweł Budzyński}
\title{}
\date{07.01.2017r.}

\begin{document}
	\maketitle

		
	\section{Wstęp}
	
	\subsection{Czym jest ewolucja wszechświata?}
	Odpowiedź na to pytanie warto rozpocząć od wyjaśnienia, czym jest kosmologia. Czym zajmuje się ta dziedzina? Jakich odpowiedzi szuka?\\\\
	W najprostszym rozumieniu kosmologia jest nauką zajmującą się budową wszechświata i jego ewolucją. Ciężko doszukiwać się momentu narodzin tej dziedziny. Możemy śmiało założyć, że ludzie od zawsze stawiali pytania dotyczące słońca, gwiazd, innych planet, wszechświata. Jednak dzięki rozwojowi nauki obecna kosmologia jest pełnoprawną nauką ścisłą potrafiącą rozwiązywać precyzyjne zagadnienia.  \\\\
	Temat ewolucji wszechświata jest ekscytującym zagadnieniem zajmującym ludzi od wieków. Skąd wziął się nasz wszechświat? Gdzie miał swój początek? Jaka czeka go przyszłość? Podobne pytania można mnożyć w nieskończoność tak jak nieskończony jest wszechświat o który pytamy. Na wiele z tych pytań wciąż nie znamy odpowiedzi.
	Całe szczęście odkrycia dokonane w ciągu ostatnich lat pozwolą nam na znalezienie rozwiązań kilku problemów. \\\\
	\subsection{Motywacje i cel projektu}
	W pracy użyjemy równań opracowanych przez A.Friedmanna do poznania\\ losów wszechświata. Odpowiemy na pytania dotyczące możliwych scenariuszy ewolucji wszechświata, poznamy dynamikę ewolucji oraz spróbujemy oszacować wiek naszego wszechświata\\\\
	\textbf{Alexander A. Friedmann}(1888-1925) był rosyjskim matematykiem oraz meteorologiem. Jako pierwszy udowodnił matematycznie rozszerzanie się wszechświata. W swoich badaniach zajmował się problemami kosmologii związanymi z ewolucją wszechświata.\cite{Biog01}\\\\
	
	
	 
	\section{Hipotezy}
	\subsection{Wielki Wybuch}
	Georges Lemaitre, który równolegle z Friedmannem, prowadził badania na temat rozszerzania się wszechświata, uważał że jeśli galaktyki oddalają się od siebie to w przeszłości musiały znajdować się bardzo blisko, a ich gęstość była bardzo duża. Lemaitre stworzył dwa modele rozprzestrzeniania się wszechświata oparte na tej koncepji. Pierwszy model przedstawia, że wszechświat będzie się rozszerzał w nieskończoność. Drugi natomiast, zakłada że galaktyki oddalają się od siebie na tyle wolno, że grawitacją może zwolnić i zatrzymać ekspansję.
	Badania Friedmana pokrywają się z założeniami Lemaitrego.
	Tylko że wyniki Friedmana sugerują również trzecią możliwość, taką że  po pewnym czasie nastąpi spowolnienie ekspansji, kosmos przestanie się rozszerzać a siły które spowolniły ekspansje, spowodują że kosmos będzie się kurczyć aż do punktu.
	
	\subsection{Inflacja Wszechświata}
	Według tej teorii  po Wielkim Wybuchu przez krótką chwilę masa wszechświata miała zaledwie kilka gramów i był niewiarygodnie mały. Po czym był krótki okres ekspansji, tak zwana inflacja, co odróżnia tą hipotezę od Wielkiego Wybuchu. I w tym krótkim czasie ekspansji powstała cała materia wszechświata i rozmiary wzrosły do takich wymiarów jak przewiduje Wielki Wybuch.	
		
	\subsection{Hipoteza Stanu Stacjonarnego Wszechświata}
	Hipoteza ta była postawiona jako alternatywa teorii ekspansji wszechświata zapoczątkowanego w jednym punkcie. Według niej wszechświat jest stacjonarny, oznacza to że w każdej chwili wygląda tak samo z każdej pozycji. Nie było by początku wszechświata ani jego końca. Istniałby on zawsze i wyglądał tak jak teraz.
	
	\section{Znaczące odkrycia}
	\subsection{Kosmiczne promieniowanie tła}
	Tak zwane promieniowanie reliktowe zostało wykryte w 1965r. przez A. Penziasa i R. Wilsona. Powstały one poprzez oddzielenie się materii od promieniowana, w czasach niedługo po Wielkim Wybuchu kiedy to wszechświat stał się na tyle rzadki, że fotony promieniowania miały od jego powstania aż do teraz nikłe szanse na znalezienie cząstek materii zdolnych do ich pochłonięcia. Dzięki Fotonom tak powstałych bardzo dawno temu możemy dziś stworzyć mapę tego promieniowania z różnych stron wszechświata.	Dzięki temu możemy wykluczyć niektóre wcześniej postawione hipotezy na przykład Hipoteze Stanu Stacjonarnego
	\subsection{Widma}
	W 1924 roku Hubble  wykazał, że nasza galaktyka nie jest jedyna we wszechświecie. Zauważył również, że w widmach gwiazd widać ten sam układ kolorów co w naszej galaktyce. Z tą różnicą, że są przesunięte w kierunku czerwonego krańca widma. Potwierdza to fakt, że galagtyki się oddalają, powieważ gdyby jakieś galaktyki się przybliżały do ziemi to układy kolorów były by przesunięte ku niebieskiemu krańcowi. 
	\section{Podstawy teoretyczne}
	Istnieją dwa modele opracowane przez Friedmanna. Pierwszy z nich modeluje zmianę czynnika skali w zależności od czasu kosmicznego, w dużym uproszczeniu jest to prędkość rozszerzania wszechświata(miara tempa ekspansji). Drugie równanie, nazywane równaniem na przyspieszenie, zawiera drugą pochodną czynnika skali po czasie. Czym więc jest wspomniany czynnik skali i dlaczego odgrywa tak ważną rolę w równaniach Friedmanna?
	\subsection{Czynnik skali $a$, topologia wszechświata, krzywizna~$k$}
	Czynnik skali jest bezwymiarową wielkością, zależną od czasu kosmicznego\cite{Cosmology}, określającą odległość między dwoma punktami względem odległości we współrzędnych współporuszających się. 
	Współrzędne współporuszające się są uproszczeniem wprowadzonym na potrzeby rozwiązań problemów powiązanych z przestrzenią we wszechświecie. Dzięki temu założeniu przyjmuje się że wszechświat jest statyczny, to znaczy pominąć fakt jego ekspansji przy rozważaniu jego kształtu(topologii). Wykorzystując te uproszczenie zakładamy, że w nowym układzie współrzędnych galaktyki mają stałe w czasie położenia. \\\\
	Zakłada się istnienie trzech możliwych geometrii:
	\begin{enumerate}
		\item Krzywizna zerowa - przestrzeń płaska. Suma kątów w trójkącie wynosi $180^\circ$.
		\item Krzywizna ujemna - przestrzeń hiperboliczna. Suma kątów w trójkącie jest mniejsza od $180^\circ$.
		\item Krzywizna dodatnia - przestrzeń sferyczna. Suma kątów w trójkącie jest większa od $180^\circ$. 
	\end{enumerate}
	Obecny stan wiedzy oraz dokonane pomiary silnie sugerują, że nasz wszechświat posiada geometrię płaską.\cite{Curvature} 
	
	\subsection{Prawo Huble'a}
	Ogległość między dowolną parą punktów przyjmując izotropowe rozszerzanie, można zapisać w postaci
	\begin{equation}
	r(t)=r(t_{0})a(t)
	\end{equation}
	przy czym  R(t) nazywamy dowolną funkcję skalującą.
	Fundamentalne prawo Hubble'a otrzymujemy 
	\begin{equation}
	v=\dot{r}(t)=r(t_{0})\dot{a}(t)=r(t_{0})a(t) \frac{\dot{a}(t)}{a(t)}=\frac{\dot{a}(t)}{a(t)}r= Hr
	\end{equation}
	\begin{equation}
	v=Hr
	\end{equation}
	Równanie powyższe mówi, że galaktyki oddalają się od nas z prędkościami proporcjonalnymi do ich odległości. Znaczy to, że im dalej od nas znajduje się galaktyka, z tym większą prędkością się od nas oddala. Oddalają się one nie tylko od nas, lecz również od siebie nawzajem. Jest to tzw. ekspansja jednorodna. 
	Otrzymujemy też stąd informację taką , że stała wynosi (tylko przestrzennie)
	\begin{equation}
	H(t)=\frac{\dot{a}}{a}
	\end{equation}

	\subsection{Zasada kosmologiczna}
	Zasada kosmologiczna mówi o tym, że obserwowane, ogólne cechy Wszechświata wszędzie są jednakowe, nie są zależne od tego w jakim rejonie Wszechświata obserwator znajduje się. Zasada ta bazuje na przeświadczeniu, że my nie zajmujemy miejsca wyjątkowego.
	Zgodnie z tą zasadą we wszechświecie nie ma wyróżnionych punktów, w tym jakiegokolwiek jednego miejsca, w którym nastąpił pierwotny wybuch inicjujący rozszerzanie się wszechświata.
	Zgodnie z tą zasadą zakłada się jednorodność i izotropowość w dużych skalach przestrzennych\cite{ZasadaKosm}.\\
	Jednorodność oznacza że przestrzeń i materia we Wszechświecie są rozłożone jednorodnie. Izotropowość głosi że patrząc w dowolnym kierunku na Wszechświat widzimy tę samą ilość materii, co oznacza że galaktyki równomiernie pokrywają niebo.
	
	\section{Równianie Friedmanna}
	Rozważamy punkt materialny o masie $m$ w odległości $R(t)$ do dowolnego centrum obserwacji, oddalający się z prędkością $v$. Przyjmując, że działa na niego grawitacyjnie tylko masa $M$ kuli o promieniu $R$ oraz korzystając z Newtonowskiej teorii grawitacji, dostajemy z zasady zachowania energii: 
	\begin{equation}
		E = \frac{m v^2}{2} + \left(-\frac{GmM}{R}\right) .
	\end{equation}
	Wiedząc, że $v = \dot{R}$ i $M = \rho \frac{4}{3} \pi R^3$ otrzymujemy
	\begin{equation}
		E = \frac{1}{2}m\dot{R}^2 - \frac{4\pi}{3} G\rho R^2 m.
	\end{equation}
	W tym momencie jesteśmy już bardzo blisko otrzymania równania Friedmanna. Wystarczy tylko wprowadzić do wzoru czynnik skali Wszechświata.
	\begin{equation}
		r = a(t)\cdot x.
	\end{equation}
	Podstawiając mamy
	\begin{equation}
		E = \frac{1}{2} m \dot{a}^2 x^2 - \frac{4\pi}{3} G \rho a^2 x^2 m.
	\end{equation}
	Szukamy zmiany czynnika skali w czasie. Mnożąc stronami przez $\frac{2}{ma^2x^2}$ otrzymujemy
	\begin{equation}
		\left(\frac{\dot{a}}{a}\right) ^ 2 = \frac{8\pi G }{3}\rho + \frac{2E}{mx^2a^2}.
	\end{equation}
	Wprowadzając stałą krzywizny przestrzeni, niezmienną w czasie i przestrzeni 
	\begin{equation}
		k = - \frac{2E}{mc^2x^2}.
	\end{equation}
	Uzyskujemy pełne równanie Friedmanna
	\begin{equation}
		H^2 \equiv \left(\frac{\dot{a}}{a}\right)^2 = \frac{8\pi G}{3}\rho - \frac{k c^2}{a^2} .
	\end{equation}
	
	
	\subsection{Rozwiązania}
	Kluczem do rozwiązania równań opisujących ewolucję Wszechświata okazuję się być znajomość równania stanu $p = p(t)$. Oznacza to, że o ewolucji Wszechświata decyduje jego zawartość. Eksperymentalnie wykazano, że głównymi składnikami Wszechświata są:
	\begin{enumerate}
		\item Pył - materia nierelatywistyczna dla której $p_{m}=0$. 
		
		\item Promieniowanie - $p_{\gamma} = \frac{\rho }{3}$,
		
		\item Ciemna energia - $p_{\Lambda} = - \rho_{\Lambda}$.
	\end{enumerate}
	Podstawiając te wartości do równania ciągłości 
	\begin{equation}
	\dot{\rho} = -3H(t)(\rho+p)a,
	\end{equation}
	otrzymujemy kolejno
	\begin{equation}
	\rho_{m} \propto \frac{1}{a^3},\,\,\,\,\, \rho_{\gamma} \propto \frac{1}{a^4},\,\,\,\,\,\, \rho_{\Lambda} = const.
	\end{equation}
		
	\section{Eksperyment}
	\subsection{Dynamika ekspansji}
	Obliczmy zachowanie tempa ewolucji dla wszechświata wypełnionego pyłem oraz dla wszechświata wypełnionego promieniowania.
	\\
	Otrzymujemy kolejno:
	\begin{enumerate}
		\item Pył: 
		\begin{equation}
			H = \frac{\dot{a}}{a} = \frac{2}{3} t \implies a(t) = e^{\frac{2}{3} ln(t)} + C,
		\end{equation}
		\item Promieniowanie
		\begin{equation}
			H = \frac{\dot{a}}{a} = \frac{1}{2} t \implies a(t) = e^{\frac{1}{2} ln(t)} + C,
		\end{equation}
	\end{enumerate}
	\begin{center}
%			\includegraphics[width=15cm]{wykres2.pdf}
	\end{center}
	Na wykresie widać, że wszechświat wypełniony promieniowaniem rozszerza się o wiele wolniej niż wszechświat wypełniony pyłem. Dodatkowo można zauważyć, że tempo ekspansji maleje z czasem.

	\subsection{Los Wszechświata}
	Odpowiedzmy sobie na pytanie, czy możliwe jest zatrzymanie rozszerzania się Wszechświata.
	\begin{equation}
		H^2 = \frac{8\pi G \rho}{3} - \frac{k c^2}{a^2} \implies H = 0.
	\end{equation}
	
	Okazuje się, że wpływ na to ma współczynnik krzywizny przestrzeni $k$. Wyróżnia się trzy przypadki:
	\begin{enumerate}
		\item $k < 0 \rightarrow H$ zawsze dodatnie. Wszechświat nieustannie się rozszerza,
		\item $k = 0$, po wstawieniu $\rho_{m} \propto \frac{1}{a^3}$ mamy
		\begin{equation}
			\lim\limits_{t \rightarrow \infty} H^2 = \lim\limits_{t \rightarrow \infty} \frac{8\pi G \rho}{3} - \frac{k c^2}{a^2} = \lim\limits_{t \rightarrow \infty} \frac{1}{a^3},
		\end{equation} ekspansja postępuje coraz wolniej,
		\item $k > 0$, po wstawieniu $\rho_{m} \propto \frac{1}{a^3}$, dominującą w czasie wartością staje~się~$a$
		\begin{equation}
			\lim\limits_{t \rightarrow \infty} H^2 = \lim\limits_{t \rightarrow \infty} \frac{8\pi G \rho}{3} - \frac{k c^2}{a^2} = \lim\limits_{t \rightarrow \infty} \frac{1}{a^3} - \frac{1}{a^2}
		\end{equation}
		Pierwszy wyraz maleje znacznie szybciej niż drugi, a więc $H^2$ na początku jest dodatnie, następnie tępo wzrostu słabnie. Ostatecznie wartość $H^2$ sięga zera. Interpretujemy to jako rozszerzanie się Wszechświata które z czasem słabnie a następnie przechodzi w kurczenie się.
		
		
	\end{enumerate}
	
	
	\subsection{Gęstość krytyczna wszechświata}
	W kosmologii definiujemy to pojęcie jako średnią gęstość materii przekładającą się na Wszechświat o płaskiej geometrii i zerowej krzywiźnie. Mówiąc w uproszczeniu Wszechświat w którym funkcjonuje geometria Euklidesowa.
	\\
	Cofnijmy się do równania postaci
	\begin{equation}
		\dot{R}^2 - \frac{8 \pi G \rho}{3} R^2 = \frac{2E}{m},
	\end{equation}
	Gdzie $\frac{2E}{m} = - k c^2 $. Wiemy już że płaskiej geometrii odpowiada $k = 0$, z czego wynika że szukamy rozwiązania dla $E = 0$.
	\begin{equation}
		\dot{R}^2 - \frac{8 \pi G \rho_{kr} }{3} R^2 = 0.
	\end{equation}
	Przypomnijmy, że
	\begin{equation}
		R(t) = a(t)\cdot x \implies \left(\frac{\dot{R}}{R}\right)^{2} = \left(\frac{\dot{a } x}{a x}\right)^{2} = \left(\frac{\dot{a } }{a }\right)^{2}= H^2.
	\end{equation}
	Zatem wyprowadzając otrzymujemy ostatecznie
	\begin{equation}
		\rho_{kr} = \frac{3 H^2}{8 \pi G}.
	\end{equation}
	
	Najnowsze badania\cite{Parameters} dowodzą, że wartość określana jako $\Omega = \frac{\rho}{\rho_{kr}} \approx 0$, co pozwala nam wnioskować, że nasz Wszechświat ma krzywiznę zerową, czyli że jest płaskiej geometrii.  
	
	
	\subsection{Obliczenie wieku wszechświata}
	Okazuje się, że, niezależnie od krzywizny, według modelu Friedmanna funkcja R(t) jest zbliżonego kształtu. Wykres przedstawia poglądowy obraz sytuacji.	
	
	\begin{center}
%		\includegraphics[width=12cm]{wykres3.pdf}
	\end{center}
	
	Przyjmując $\dot{R} = \tan(\alpha) = \frac{R}{T}$, zatem $T = \frac{R}{\dot{R}} = H^{-1}$. Wielkość $T$ nazywamy czasem Hubble'a.
	
	Oczywistym jest, że czas obecny $t_{0}$ jest mniejszy od wieku wszechświata~$T$.
	
	Znając aktualne przybliżenie stałej Hubble'a $H = 67,15 (km/s)/Mpc$ ~\cite{stalaHubble}, uzyskujemy ograniczenie
	\begin{equation}
		t_{0} < T = H^{-1} \approx 15\cdot 15^9 lat.
	\end{equation}
	Sprawdzając dla poprzednich szacowań stałej Hubble'a ~\cite{staraHubble}, mieszczących się między $60 (km/s)/Mpc$ a $80 (km/s)/Mpc$, dostajemy szersze szacowanie
	\begin{equation}
		12.6\cdot10^9 lat\,\, < t_{0} < 1.68\cdot 10^9 lat
	\end{equation}
	
	
	
	
	\section{Podsumowanie}
	

	
	\bibliography{mybib}
	\bibliographystyle{plain}
\end{document}